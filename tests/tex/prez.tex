
% Файл proba2.tex 

% 

% Преамбула – в ней содержаться основные сведения о презентации

% 

\documentclass{beamer} 
\usepackage{beamerthemesplit} 

% % Стили презентации в данном пакете задаются именем Университета, в

стиле и цветах которого вы хотите оформить презентации

% 

\usetheme{CambridgeUS} 

% Выпишем часть возможных стилей, некоторые из них могут содержать

% дополнительные опции

% default, Bergen, Madrid, AnnArbor,Pittsburg, Rochester, 

% Antiles, Montpellier, Berkley, Berlin 

% 

% Далее пакеты, необходимые вам для создания презентации

\usepackage{amsmath} 

\usepackage[english]{babel} 

% 

% Если у вас есть логотип вашей кафедры, факультета или университета, то

% его можно включить в презентацию. 

% Необходимо наличие графического файла в текущей директории !!!

%\Logo{\includegraphics[width=1cm]{logo.eps}} 

% 

% Далее начинается сама презентация

%

\begin{document} 

% 

% Первый слайд

\frame{ 

\frametitle{First slide} 

Hello World. 

} 

% Все очень просто. 

%Переходы между слайдами задаются командами

% \trans***<overlay specification>[options] 

% Здесь

% <overlay specification> - указание к какой части слайда применить

% переход; 

% [options] – указание длительности и направления перехода

% 

%Вместо *** нужно подставить – dissolve, blindshorizontal, blindsvertical,

% boxin, boxout, glitter, splitverticalin, splitverticalout,wipe, duration и т.д. 

% 

% Кроме этого, можно задавать свои способу перехода – см. документы

% Например

\begin{frame} \transwipe[direction=90] 

% \transdissolve[duration=0.2] 

\frametitle{Second slide} 

It’s my second slide. 

\end{frame}

% 

% Напомним, что переходы видны только при полноэкранном режиме

% работы Adobe Reader. 

% 

% Рассмотрим более сложный слайд с переходами

% 

\begin{frame} 

\frametitle{The quest for $\pi$} 

\begin{itemize} 

\item The following formula computes $8$ correct digits per iteration 

(Ramanujan): \pause 

\item 

\begin{small} 

\begin{equation*} 

\frac{1}{\pi}=\sum_{n=0}^\infty 

\frac{(\frac{1}{4})_n(\frac{2}{4})_n(\frac{3}{4})_n}{n!^3} 

\bigl(2\sqrt{2}(1103+26390n)\bigr)\frac{1}{(99^2)^{2n+1}} 

\end{equation*} 

\end{small} 

\end{itemize} 

\transglitter<1>[direction=315] 

\transboxin<2>[direction=90] 

\end{frame} 

% Как и ранее для различных цветов используется команда \color{цвет}, для

% переходов команда \pause. Стиль перехода задается либо в преамбуле, 

 % либо может быть задан в заголовке слайда и этих стилей огромное

% количество.

% 

% Красивые переходы нужны для того, чтобы переключить внимание

% слушателей, обычно используются для подчеркивания мысли, для

% пробуждения уставших слушателей или для вешания лапши на уши

% некомпетентному начальству. Применять при докладах среди

% специалистов рекомендуется в очень ограниченных дозах, так как для них

% важнее не ваши компьютерные навыки, а содержание презентации. 
% 

% Более сложный структурированный пример

\frame{ 

\frametitle{Light Scheme} 

\begin{itemize} 

\item A first item \pause 

\item A second item\pause 

\begin{itemize} 

\item A sub item\pause 

\begin{itemize} 

\item \color{blue} A sub sub item 

\item \color{red} Another sub sub item 

\end{itemize} 

\item Back to sub item \pause 

\end{itemize} 

\item Last item \pause 

\end{itemize} 

} 

% 

% Теперь приведем пример с изменением прозрачности текста и

%разбиением слайда на блоки

%

\frame { 

\frametitle{Theorem} 

\begin{block}{First}<1> 

The weak cardinality theorems hold both for recursion and automata 

theory \alert{by coincidence}. 

\end{block} 

\begin{block}{Second Explanation}<1-2> 

The weak cardinality theorems hold both for 

recursion and automata theory, \alert{because they are 

instantiations of\\ single, unifying theorems}. 

\end{block} 

\vskip1em 

\visible<2->{ 

The second explanation is correct.\\ 

The theorems can (almost) be unified using first-order logic. 

} 

} % 

% Пакет Beamer почти идеален для включения анимации в вашу

% презентацию – музыка, кино и другие мультимедиа файлы легко

% присоединяются к вашему файлу. Кроме этого, анимацию можно создать и

%своими руками в TeX’e.

% 

% Анимация – пример “летающего” слайда. Обратите внимание на

% изменение команд в начале и конце слайда ( ранее было \frame{ }) 

% 

\newcount\opaqueness 

\begin{frame} 

\animate<2-10> 

\animatevalue<1-10>{\opaqueness}{100}{0} 

\begin{colormixin}{\the\opaqueness!averagebackgroundcolor} 

\frametitle{Fadeout Frame} 

This text (and all other frame content) will fade out when the 

second slide is shown. This even works with 

{\color{green!90!black}colored} \alert{text}. 

\end{colormixin} 

\end{frame} 

% 

\newcount\opaqueness 

\newdimen\offset 

\begin{frame} 

\frametitle{Flying Theorems (You Really Shouldn't!)} 

\animate<2-14> 

\animatevalue<1-15>{\opaqueness}{100}{0} 

\animatevalue<1-15>{\offset}{0cm}{-5cm} 

\begin{colormixin}{\the\opaqueness!averagebackgroundcolor} 

\hskip\offset 

\begin{minipage}{\textwidth} 

\begin{theorem} 

This theorem flies out. 

\end{theorem} 

\end{minipage} 

\end{colormixin} 

\animatevalue<1-15>{\opaqueness}{0}{100} 

\animatevalue<1-15>{\offset}{-5cm}{0cm} 

\begin{colormixin}{\the\opaqueness!averagebackgroundcolor} \hskip\offset 

\begin{minipage}{\textwidth} 

\begin{theorem} 

This theorem flies in. 

\end{theorem} 

\end{minipage} 

\end{colormixin} 

\end{frame} 

% 

% На этом краткое знакомство с пакетом Beamer заканчивается. 

%

\end{document} 

% 

% Конец файла

