
\documentclass{beamer} 
\usepackage{beamerthemesplit} 

\usetheme{Madrid}%Frankfurt} 
%Boadilla}
\usecolortheme{default}

\setbeamertemplate{footline}{%
    \hspace{0.94\paperwidth}%
    \usebeamerfont{title in head/foot}%
    \large{\insertframenumber}%
}

\setbeamertemplate{navigation symbols}{}

% Выпишем часть возможных стилей, некоторые из них могут содержать

% дополнительные опции

% default, Bergen, Madrid, AnnArbor,Pittsburg, Rochester, 

% Antiles, Montpellier, Berkley, Berlin, CambridgeUS

% 

% Далее пакеты, необходимые вам для создания презентации
\usepackage[T2A]{fontenc}
\usepackage[utf8]{inputenc}
\usepackage[english,russian]{babel}
\usepackage{amssymb,amsfonts,amsmath,mathtext}
\usepackage{cite,enumerate,float,indentfirst}
\usepackage{graphicx}

%\DeclareSymbolFont{letters}{OT1}{cmr}{m}{n}

\usefonttheme[stillsansserifsmall]{serif}
%\usefonttheme[onlymath]{serif}

\title{
Отчет о профессиональной деятельности
за межаттестационный период 2011-2013}
%МАТЕМАТИЧЕСКОЕ МОДЕЛИРОВАНИЕ УПРУГИХ МАКРОХАРАКТЕРИСТИК 
%ДЛЯ КОМПОЗИТНЫХ МАТЕРИАЛОВ С ВКЛЮЧЕНИЯМИ В ВИДЕ ПРЯМЫХ 
%ПЕРИОДИЧЕСКИ РАСПРЕДЕЛЁННЫХ ПАРАЛЛЕЛЬНЫХ ВОЛОКОН}
\author{А.Ф. Власко, инженер-программист}
\institute{Научная лаборатория математического моделирования в строительстве и промышленности}
\date{09.10.2013}


\begin{document} 

\maketitle

\setbeamertemplate{frametitle}[default][center]

\frame{ \frametitle{Инженерные рассчеты} 

\begin{figure}
    \begin{center}
        \includegraphics[scale=0.5]{eng}
    \end{center}
\end{figure}

}

\frame{ \frametitle{CAE} 

\begin{figure}
    \begin{center}
        \includegraphics[scale=0.5]{all_logo}
    \end{center}
\end{figure}

}

\frame{ \frametitle{Композитные материалы} 

\begin{figure}
    \begin{center}
        \includegraphics[scale=0.5]{comp}
    \end{center}
\end{figure}

}

\frame{ \frametitle{2-периодическая среда} 

\begin{figure}
    \begin{center}
        \includegraphics[scale=0.2]{cube}
    \end{center}
\end{figure}

}

\frame{ \frametitle{Инструменты} 

\begin{figure}
    \begin{center}
        \includegraphics[scale=0.5]{tools}
    \end{center}
\end{figure}

}

\frame{ \frametitle{Вычислительное ядро} 

\begin{figure}
    \begin{center}
        \includegraphics[scale=0.5]{core}
    \end{center}
\end{figure}

}

\frame{ \frametitle{Распределение нагрузки (первое главное напряжение)
в одной ячейке} 

\begin{figure}
    \begin{center}
        \includegraphics[scale=0.2]{main_stress_1}
    \end{center}
\end{figure}

}

\frame{ \frametitle{Приблизительно 37000 строк кода} 

\begin{figure}
    \begin{center}
        \includegraphics[scale=0.2]{code}
    \end{center}
\end{figure}

}

\frame{ \frametitle{Участие на конференциях и публикации} 
\begin{itemize}
    \item \footnotesize
Участие с докладом во всероссийской  конференции «Механика микронеоднородных материалов и разрушение», Екатеринбург, 23-27.04.2012.
    \item
Участие с докладом во всероссийской  конференции молодых ученых «Наука и инновации XXI века», Сургут, 28.11.2012.
    \item
Участие с докладом во VI всероссийской  научно-технической конференции «Актуальные вопросы строительства», Новосибирск, 9-11.04.2013.
    \item
Участие во XXIII всероссийской конференции «Численные методы решения задач теории упругости и пластичности», Барнаул, 28-29.06.2013.
    \item
Публикация в издании из списка ВАК «Вестник СибАДИ»: «Математическое моделирование макрохарактеристик процесса теплопроводности для волокнистых материалов при расчете строительных конструкций на действие тепловых нагрузок», Вестник СибАДИ. – 2012. – №3(25). – С.69-74.
    \item
Публикация в издании из списка ВАК «Вестник СибАДИ»: «Математическое моделирование упругих макрохарактеристик для волокнистых материалов при расчете конструкций транспортных сооружений», Вестник СибАДИ. – 2013. – №1(29). – С.58-64.
\end{itemize}
}

\end{document} 

\frame{ \frametitle{Стационарное уравнение равновесия} 

\begin{equation}
    \begin{array}{c}
        \mathrm{\frac{\partial \sigma_{\alpha x}}{\partial x} +
        \frac{\partial \sigma_{\alpha y}}{\partial y} + 
        \frac{\partial \sigma_{\alpha z}}{\partial z} + F_{\alpha} = 0,} \\ \\
        \mathrm{\sigma_{\alpha \beta} = \sum\limits_{\varphi,\psi \in \{x,y,z\}} 
        E_{\alpha \beta \varphi \psi} 
        \frac{\partial u_{\varphi}}{\partial \psi}} \\ \\ 
        \mathrm{\alpha, \beta \in \{x,y,z\}}
    \end{array} 
\end{equation}

\begin{equation}
    \mathrm{[\sigma_{\alpha n}] = 0, [u_{\alpha}] = 0}
\end{equation}

\begin{equation}
        \mathrm{[\sigma_{\alpha n}] = 
        [\sigma_{\alpha x}]n_x +
        [\sigma_{\alpha y}]n_y +
        [\sigma_{\alpha z}]n_z}
\end{equation}

} 


\frame{ \frametitle{Стационарное уравнение равновесия в безразмерных величинах} 

\begin{equation}
    \begin{array}{c}
        \mathrm{\frac{\partial \sigma_{\alpha x}}{\partial x}\varepsilon +
        \frac{\partial \sigma_{\alpha y}}{\partial y}\varepsilon + 
        \frac{\partial \sigma_{\alpha z}}{\partial z}\varepsilon + F_{\alpha} = 0,} \\ \\
        \mathrm{\sigma_{\alpha \beta} = \sum\limits_{\varphi,\psi \in \{x,y,z\}} 
        E_{\alpha \beta \varphi \psi} 
        \frac{\partial u_{\varphi}}{\partial \psi} \varepsilon} \\ \\ 
        \mathrm{\alpha, \beta \in \{x,y,z\}}
    \end{array} 
\end{equation}

\begin{equation}
    \begin{array}{c}
    \mathrm{x \leftrightarrow \frac{x}{L},
    y \leftrightarrow \frac{y}{L},
    z \leftrightarrow \frac{z}{L},
    u_{\alpha} \leftrightarrow \frac{u_{\alpha}}{u^*},} \\ \\
    \mathrm{E_{\alpha \beta \varphi \psi} \leftrightarrow  
    \frac{E_{\alpha \beta \varphi \psi}}{E^*},
    \sigma_{\alpha \beta} \leftrightarrow \frac{\sigma_{\alpha \beta}}{\sigma^*},
    F_{\alpha} \leftrightarrow \frac{F_{\alpha}h}{\sigma^*},} \\ \\
    \mathrm{\sigma^*=\frac{E^*u^*}{h}}
    \end{array} 
\end{equation}

\begin{equation*}
    \mathrm{\varepsilon = \frac{h}{L} \ll 1}
\end{equation*}

}

%\frame{ \frametitle{Одна ячейка} 

%\begin{figure}
%    \begin{center}
%        \includegraphics[scale=0.3]{cell}
%    \end{center}
%\end{figure}

%}

\frame{ \frametitle{Верхняя и нижняя оценки Рейсса-Фойгта} 

\begin{figure}
    \begin{center}
        \includegraphics[scale=0.2]{cell}
    \end{center}
\end{figure}

\begin{equation}
    \frac{1}{\frac{\Theta^I}{E_{\alpha\beta\varphi\psi}^I} +
    \frac{\Theta^B}{E_{\alpha\beta\varphi\psi}^B}} \le
    \widetilde{E}_{\alpha\beta\varphi\psi} \le
    \Theta^IE_{\alpha\beta\varphi\psi}^I +
    \Theta^BE_{\alpha\beta\varphi\psi}^B
\end{equation}

\begin{equation}
    \begin{array}{c}
        \Theta^I - \text{доля включения (коэффициент армирования)} \\ \\
        \Theta^B - \text{доля связующего}
    \end{array} 
\end{equation}

}

\frame{ \frametitle{Верхняя и нижняя оценки Хашина-Штрикмана} 

\begin{equation}
    \begin{array}{c}
    K^B + \frac{\Theta^I}{1/(K^I-K^B) + \Theta^B/(K^B+G^B)} \le 
    \widetilde{K}_{xy} \le \\
    K^I + \frac{\Theta^B}{1/(K^B-K^I) + \Theta^I/(K^I+G^I)}
    \end{array} 
\end{equation}

\begin{equation}
    \begin{array}{c}
    G^B + \frac{\Theta^I}{1/(G^I-G^B) + \Theta_B(K^B+2G^B)/(2G^B(K^B+G^B))} \le 
    \widetilde{G}_{xy} \le \\
    G^I + \frac{\Theta^B}{1/(G^B-G^I) + \Theta_I(K^I+2G^I)/(2G^I(K^I+G^I))} \le 
    \end{array} 
\end{equation}

\begin{equation}
    \begin{array}{c}
    G^B + \frac{\Theta^I}{1/(G^I-G^B) + \Theta^B/(2G^B))} \le 
    \widetilde{G}_{xz} \le \\ 
    G^I + \frac{\Theta^B}{1/(G^B-G^I) + \Theta^I/(2G^I)} 
    \end{array} 
\end{equation}

\begin{equation}
    \begin{array}{c}
    \frac{\Theta^I\Theta^B}{\Theta^I/K^B+\Theta^B/K^I+1/G^B} \le 
    \frac{\widetilde{E}_{zz} - \Theta^IE^I - \Theta^BE^B}{4(\nu^I-\nu^B)^2} \le \\ 
    \frac{\Theta^I\Theta^B}{\Theta^I/K^B+\Theta^B/K^I+1/G^I} 
    \end{array} 
\end{equation}

\begin{equation}
    \begin{array}{c}
    \frac{\Theta^I\Theta^B}{\Theta^I/K^B+\Theta^B/K^I+1/G^B} \le 
    \frac{\widetilde{\nu}_{xz} - \Theta^I\nu^I - \Theta^B\nu^B}
    {(\nu^I-\nu^B)(1/K^B-1/K^I)} \le \\ 
    \frac{\Theta^I\Theta^B}{\Theta^I/K^B+\Theta^B/K^I+1/G^I} 
    \end{array} 
\end{equation}

}

\frame{ \frametitle{Выражения Хашина-Хилла для полидисперсной модели} 

\begin{equation}
    \begin{array}{c}
    \widetilde{K}_{xy} = k^B + \frac{G^B}{3} + \\
    +\frac{\Theta^I}{1/[k^I-k^B+1/3(G^I-G^B)]+(1-\Theta^I)/(k^B+4/3G^B)}
    \end{array} 
\end{equation}

\begin{equation}
    \begin{array}{c}
    \widetilde{G}_{xz} = 
    \frac{G^I(1+\Theta^I)+G^B(1-\Theta^I)}
    {G^I(1-\Theta^I)+G^B(1+\Theta^I)} G^B
    \end{array} 
\end{equation}

\begin{equation}
    \begin{array}{c}
    \widetilde{E}_{zz} = \Theta^IE^I+\Theta^BE^B+ \\
    +\frac{4\Theta^I\Theta^B(\mu^I-\nu^B)^2G^B}
    {\Theta^BG^B/(k^I-G^I/3)+\Theta^IG^B/(k^B-G^B/3)+1}
    \end{array} 
\end{equation}

\begin{equation}
    \begin{array}{c}
        \widetilde{\nu}_{xz} = \Theta^I\nu^I + \Theta^B\nu^B + \\
        +\frac
        {\Theta^I\Theta^B(\nu^I-nu^B)(G^B/(k^B-G^B/3)-G^B/(k^I-G^I/3))}
        {\Theta^BG^B/(k^I-G^I/3)+\Theta^IG^B/(k^B-G^B/3)+1}
    \end{array} 
\end{equation}

}

\frame{ \frametitle{Асимптотическое разложение} 

\begin{equation}
\mathrm{u_{\alpha}^{(n)} = v_{\alpha}^{(n)} + \sum\limits_{\varphi \in \{x,y,z\}}
\sum\limits_{k=1}^{n}\sum\limits_{k_x+k_y+k_z=k}\left( 
\left(U_{\beta}^{v_{\varphi}}\right)^{\bar{k}}
\frac{\partial^k v_{\varphi}^{(n)}}{\partial \bar{r}^{\bar{k}}}\varepsilon^k \right)}
\end{equation}

\begin{equation}
\mathrm{\sigma_{\alpha\beta}^{(n)} = \sum\limits_{\varphi \in \{x,y,z\}}
\sum\limits_{k=1}^{n}\sum\limits_{k_x+k_y+k_z=k}\left( 
\left(\tau_{\alpha\beta}^{v_{\varphi}}\right)^{\bar{k}}
\frac{\partial^k v_{\varphi}^{(n)}}{\partial \bar{r}^{\bar{k}}}\varepsilon^k \right)}
\end{equation}

\begin{equation}
    \begin{array}{c}
        \mathrm{\bar{k}=\{k_x,k_y,k_z\} =
        k_x\bar{\text{э}}_x+k_y\bar{\text{э}}_y+k_z\bar{\text{э}}_z,} \\
        \mathrm{|\bar{k}|=k=k_x+k_y+k_z,} \\ 
        \mathrm{\partial \bar{r}^{\bar{k}}= \partial x^{k_x} \partial y^{k_y} \partial z^{k_z}}, \\
        \mathrm{k_{\alpha} \ge 0, k_{\alpha} \in \mathbb{Z}}, \\
\alpha \in \{x,y,z\}
    \end{array} 
\end{equation}

}

\frame{ \frametitle{Ячейковая система координат} 

\begin{figure}
    \begin{center}
        \includegraphics[scale=0.2]{cell_xi}
    \end{center}
\end{figure}

\begin{equation}
    \mathrm{\xi_x,\xi_y \in [0,1]}
\end{equation}

\begin{equation}
    \mathrm{x = x_i + \xi_x\varepsilon, \
    y = y_i + \xi_y\varepsilon, \
z = z.}
\end{equation}

}

\frame{ \frametitle{Краевые задачи на ячейке} 

\begin{equation}
    \begin{array}{c}
        \mathrm{\frac
        {\partial 
        \left(\tau_{\alpha x}^{v_{\mu}}\right)^{\overline{\text{э}}_{\lambda}}}
        {\partial x} +
        \frac
        {\partial
        \left(\tau_{\alpha y}^{v_{\mu}}\right)^{\overline{\text{э}}_{\lambda}}}
        {\partial y} = 0} \\ \\
        \mathrm{\left(\tau_{\alpha \beta}^{v_{\eta}}\right)^{\overline{\text{э}}_{\lambda}}=
        E_{\alpha \beta \eta \lambda} + 
        \sum\limits_{\varphi,\psi \in \{x,y\}} E_{\alpha \beta \varphi \psi} 
        \frac{\partial \left(U_{\varphi}^{v_{\eta}}\right)^{\overline{\text{э}}_{\lambda}}}
        {\partial \xi_{\psi}}} + \\
        \mathrm{+ \sum\limits_{\psi \in \{x,y\}} E_{\alpha \beta z \psi} 
        \frac{\partial \left(U_{z}^{v_{\eta}}\right)^{\overline{\text{э}}_{\lambda}}}
        {\partial \xi_{\psi}}} \\ 
        \mathrm{\alpha \in \{x,y\}, \ \eta, \lambda \in \{x,y,z\}}
    \end{array} 
\end{equation}

\begin{equation}
    \begin{array}{c}
        \mathrm{\left. \left(U_{\varphi}^{v_{\eta}}\right)^{\overline{\text{э}}_{\lambda}} 
        \right|_{\xi_{\gamma} = 0} =
        \left. \left(U_{\varphi}^{v_{\eta}}\right)^{\overline{\text{э}}_{\lambda}} 
        \right|_{\xi_{\gamma} = 1}} \\ \\
        \mathrm{\left.\left(\tau_{\alpha y}^{v_{\eta}}\right)^{\overline{\text{э}}_{\lambda}}
        \right|_{\xi_{\gamma} = 0} =
        \left.\left(\tau_{\alpha y}^{v_{\eta}}\right)^{\overline{\text{э}}_{\lambda}}
        \right|_{\xi_{\gamma} = 1}} \\ 
        \mathrm{\gamma \in \{x, y\}}
    \end{array} 
\end{equation}

}

\frame{ %\frametitle{Noname} 

\begin{equation}
        \mathrm{\left[ \left(U_{\varphi}^{v_{\eta}}\right)^{\overline{\text{э}}_{\lambda}} \right] = 0,
        \left[\left(\tau_{\alpha n}^{v_{\eta}}\right)^{\overline{\text{э}}_{\lambda}}
        \right] = 0}
\end{equation}

\begin{equation}
        \mathrm{\left \langle
        \left(U_{\varphi}^{v_{\eta}}\right)^{\overline{\text{э}}_{\lambda}}
    \right \rangle = 0}
\end{equation}

\begin{equation}
    \mathrm{\langle \_ \rangle = \int\limits_0^1\int\limits_0^1\_d\xi_xd\xi_y}
\end{equation}

\begin{equation}
    \begin{array}{c}
        \mathrm{\left(U_{\varphi}^{v_{\eta}}\right)^{\overline{\text{э}}_{\lambda}} 
         =
         \left(U_{\varphi}^{v_{\lambda}}\right)^{\overline{\text{э}}_{\eta}} }
        \\ \\
        \mathrm{\left(\tau_{\alpha n}^{v_{\eta}}\right)^{\overline{\text{э}}_{\lambda}}
         =
         \left(\tau_{\alpha n}^{v_{\lambda}}\right)^{\overline{\text{э}}_{\eta}}}
    \end{array} 
\end{equation}

}

\frame{ \frametitle{Вычисление упругих констант макросреды}

\begin{equation}
    \begin{array}{c}
        \mathrm{\widetilde{E}_{\alpha \beta \eta \lambda} =
        \left \langle 
        \left(\tau_{\alpha y}^{v_{\eta}}\right)^{\overline{\text{э}}_{\lambda}}
    \right \rangle =} \\ \\
        \mathrm{= \langle 
        E_{\alpha \beta \eta \lambda} 
        \rangle +
        \left\langle 
        \sum\limits_{\varphi,\psi \in \{x,y\}} E_{\alpha \beta \varphi \psi} 
        \frac{\partial \left(U_{\varphi}^{v_{\eta}}\right)^{\overline{\text{э}}_{\lambda}}}
        {\partial \xi_{\psi}} \right\rangle +}  \\ \\
        \mathrm{+ \left\langle \sum\limits_{\psi \in \{x,y\}} E_{\alpha \beta z \psi} 
        \frac{\partial \left(U_{z}^{v_{\eta}}\right)^{\overline{\text{э}}_{\lambda}}}
        {\partial \xi_{\psi}}  
    \right\rangle} \\ \\
        \mathrm{\alpha, \beta, \eta, \lambda \in \{x,y,z\}}
    \end{array} 
\end{equation}

}

\frame{ \frametitle{Периодические ячейки с различными формами поперечных сечений
арматурных волокон} 

\begin{figure}
    \begin{center}
        \includegraphics[scale=0.3]{cells}
    \end{center}
\end{figure}

\begin{equation}
    \begin{array}{c}
        \mathrm{w = b - \sqrt{b^2 - S^I}, b \approx 0.75} \\ \\
        \mathrm{S^I - \text{Площадь включения}}
    \end{array} 
\end{equation}

}

\end{document}

\frame{ \frametitle{Модуль Юнга $E_x$} 

\begin{figure}
    \begin{center}
        \includegraphics[scale=0.3]
        {../elastic_test_on_cell/sources/i_10/article_nowosib/10_Ex}
    \end{center}
\end{figure}

}

\frame{ \frametitle{Модуль сдвига $G_{xy}$} 

\begin{figure}
    \begin{center}
        \includegraphics[scale=0.3]
        {../elastic_test_on_cell/sources/i_10/article_nowosib/10_Mxy}
    \end{center}
\end{figure}

}

\frame{ \frametitle{Модуль сдвига $G_{xz}$} 

\begin{figure}
    \begin{center}
        \includegraphics[scale=0.3]
        {../elastic_test_on_cell/sources/i_10/article_nowosib/10_Mxz}
    \end{center}
\end{figure}

}

\frame{ \frametitle{Коэффициент Пуасона $\nu_{xy}$} 

\begin{figure}
    \begin{center}
        \includegraphics[scale=0.3]
        {../elastic_test_on_cell/sources/i_10/article_nowosib/10_Nxy}
    \end{center}
\end{figure}

}

\frame{ \frametitle{Коэффициент Пуасона $\nu_{xz}$} 

\begin{figure}
    \begin{center}
        \includegraphics[scale=0.3]
        {../elastic_test_on_cell/sources/i_10/article_nowosib/10_Nxz}
    \end{center}
\end{figure}

}

\frame{ \frametitle{Коэффициент Пуасона $\nu_{zx}$} 

\begin{figure}
    \begin{center}
        \includegraphics[scale=0.3]
        {../elastic_test_on_cell/sources/i_10/article_nowosib/10_Nzx}
    \end{center}
\end{figure}

}

\end{document} 
