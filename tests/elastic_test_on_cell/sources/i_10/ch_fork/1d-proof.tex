\documentclass[a4paper,12pt]{article}
\usepackage{graphicx}
\usepackage[utf8]{inputenc}
\usepackage[russian]{babel}
\usepackage{amsmath,amssymb}
\usepackage{tikz}
\usepackage{pgfplots}
\pgfplotsset{compat=1.7}

\makeatletter
\newcommand{\rmnum}[1]{\romannumeral #1}
\newcommand{\Rmnum}[1]{\expandafter\@slowromancap\romannumeral #1@}
\makeatother

\begin{document}

\section{Одномерное уравнение теплопроводности.}

\begin{equation} \label{e1}
    \begin{array}{c}
        \frac{d}{dx} \big( \lambda(x) \frac{dT(x)}{dx} \big) = 0, \\ \\
        \left. T(x) \right|_{x=0} = T(0), \left. T(x) \right|_{x=L} = T(L). \\ \\
        x_1 = \frac{L}{2}
    \end{array} 
\end{equation}

$x_1$ - точка разделения двух сред $\lambda_1$ и $\lambda_2$.

\begin{equation*}
    \begin{cases}
        \lambda(x) = \lambda_1; \ 0 \le x \le x_1\\
        \lambda(x) = \lambda_2; \ x_1 \le x \le L\\
    \end{cases} 
\end{equation*}

После обезразмеривания, уравнение принимает вид:

\begin{equation} \label{e1b}
    \begin{array}{c}
        \frac{d}{dx} \big( \lambda(x) \frac{dT(x)}{dx} \big) = 0, \\ \\
        \left. T(x) \right|_{x=0} = T(0), \left. T(x) \right|_{x=1} = T(1). \\ \\
         x_1 = \frac{1}{2} \\ \\
         \varepsilon = 1
    \end{array} 
\end{equation}

\subsection{Аналитическое решение.}

\begin{equation*}
        \frac{dT(x)}{dx} = \frac{C}{\lambda(x)}
\end{equation*}

\begin{equation*}
    \begin{cases}
        T(x)^{\Rmnum{1}} - T(0) = \int\limits_0^x \frac{C}{\lambda_1}dt; \ 0 \le x \le \frac{1}{2}\\
        T(1) - T(x)^{\Rmnum{2}} = \int\limits_x^1 \frac{C}{\lambda_2}dt; \ \frac{1}{2} \le x \le 1\\
    \end{cases} 
\end{equation*}

\begin{equation*}
    \begin{cases}
        T^{\Rmnum{1}}(x) = \frac{C}{\lambda_1}x + T(0); \ 0 \le x \le \frac{1}{2}\\ \\ 
        T^{\Rmnum{2}}(x) = \frac{C}{\lambda_2}(x-1) + T(1); \ \frac{1}{2} \le x \le 1\\ 
    \end{cases} 
\end{equation*}

\begin{equation*}
    T^{\Rmnum{1}}(1/2) = T^{\Rmnum{2}}(1/2)
\end{equation*}

\begin{equation*}
    \frac{C}{\lambda_1}\frac{1}{2} + T(0) = \frac{C}{\lambda_2}(\frac{1}{2}-1) + T(1)
\end{equation*}

\begin{equation}\label{e2}
    C = \frac{T(1)-T(0)}{\frac{1/2}{\lambda_1} + \frac{1-1/2}{\lambda_2}} =
    2\frac{\lambda_1 \lambda_2}{\lambda_1 + \lambda_2}(T(1)-T(0))
\end{equation}

Если $T(0) = b$ и $T(1) = a+b$, то

\begin{equation}\label{e3}
    C = 2a \frac{\lambda_1 \lambda_2}{\lambda_1 + \lambda_2}
\end{equation}

\begin{equation}\label{e4}
    \begin{cases}
        T^{\Rmnum{1}}(x) = \frac{2a\lambda_2}{\lambda_1 + \lambda_2}x + b; \ 0 \le x \le \frac{1}{2}\\ \\
        T^{\Rmnum{2}}(x) = \frac{2a\lambda_1}{\lambda_1 + \lambda_2}(x-1) + a + b; \ \frac{1}{2} \le x \le 1\\
    \end{cases} 
\end{equation}

(\ref{e4}) есть аналитическое решение уравнения (\ref{e1b}).

\subsection{Асимптотическое решение.}

\begin{equation} \label{e5}
    \begin{array}{c}
        \frac{d}{d\xi} \big( \lambda(\xi) \frac{d\Psi(\xi)}{d\xi} \big) = 
        - \frac{d\lambda(\xi)}{d\xi}, \\ \\
        \left. \Psi(\xi) \right|_{\xi=0} = \left. \Psi(\xi) \right|_{\xi=1}. \\ \\
        \left. k(\xi) \right|_{\xi=0} = \left. k(\xi) \right|_{\xi=1}. \\ \\
        k(\xi) = -(\lambda(\xi) \frac{d\Psi(\xi)}{d\xi} + \lambda(\xi)) \\ \\
        \varepsilon = 1, \xi_1 = \frac{1}{2} \\ \\
        T(0) = 0
    \end{array} 
\end{equation}

\begin{equation*}
        \frac{d\Psi(x)}{dx} = \frac{C}{\lambda(x)} - 1
\end{equation*}

\begin{equation*}
    \begin{cases}
        \Psi^{\Rmnum{1}}(x) = \int\limits_0^x (\frac{C}{\lambda_1} - 1)dt; \ 0 \le x \le \frac{1}{2}\\
        - \Psi^{\Rmnum{2}}(x) = \int\limits_x^1 (\frac{C}{\lambda_2} - 1)dt; \ \frac{1}{2} \le x \le 1\\
    \end{cases} 
\end{equation*}

\begin{equation*}
    \begin{cases}
        \Psi^{\Rmnum{1}}(\xi) = (\frac{C}{\lambda_1} - 1)\xi; \ 0 \le \xi \le \frac{1}{2}\\
        \Psi^{\Rmnum{2}}(\xi) = (\frac{C}{\lambda_2} - 1)(\xi - 1); \ \frac{1}{2} \le \xi \le 1\\
    \end{cases} 
\end{equation*}

\begin{equation*}
    \Psi^{\Rmnum{1}}(1/2) = \Psi^{\Rmnum{2}}(1/2)
\end{equation*}

\begin{equation*}
    (\frac{C}{\lambda_1} - 1)1/2 = (\frac{C}{\lambda_2} - 1)(1/2 - 1)
\end{equation*}

\begin{equation}\label{e6}
    C = \frac{1}{\frac{1/2}{\lambda_1} + \frac{1-1/2}{\lambda_2}} =
    2\frac{\lambda_1 \lambda_2}{\lambda_1 + \lambda_2}
\end{equation}

\begin{equation} \label{e7}
    \begin{cases}
        \Psi^{\Rmnum{1}}(\xi) = (\frac{2\lambda_2}{\lambda_1 + \lambda_2} - 1)\xi; \ 0 \le \xi \le \frac{1}{2}\\
        \Psi^{\Rmnum{2}}(\xi) = (\frac{2\lambda_1}{\lambda_1 + \lambda_2} - 1)(\xi - 1); \ \frac{1}{2} \le \xi \le 1\\
    \end{cases} 
\end{equation}

\begin{equation} \label{e8}
    \stackrel{o}{T}(x) = T_0(x) + \Psi(\xi)T'_0(x)
\end{equation}

$\stackrel{o}{T}(x)$ - асимптотическое приближение к $T(x)$.
Так как $\varepsilon = 1$, то $\xi = x$.
Из (\ref{e1}) $T_0(x) = ax+b$, $T'_0(x) = a$.

\begin{equation*}
    \begin{cases}
        \stackrel{o}{T^{\Rmnum{1}}}(x) = 
        ax+b + a(\frac{2\lambda_2}{\lambda_1 + \lambda_2} - 1)x;
        \ 0 \le x \le \frac{1}{2}\\
        \stackrel{o}{T^{\Rmnum{2}}}(x) = 
        ax+b + a(\frac{2\lambda_1}{\lambda_1 + \lambda_2} - 1)(x - 1);  
        \ \frac{1}{2} \le x \le 1\\
    \end{cases} 
\end{equation*}

\begin{equation} \label{e9}
    \begin{cases}
        \stackrel{o}{T^{\Rmnum{1}}}(x) = 
        \frac{2a\lambda_2}{\lambda_1 + \lambda_2}x + b;
        \ 0 \le x \le \frac{1}{2}\\
        \stackrel{o}{T^{\Rmnum{2}}}(x) = 
        \frac{2a\lambda_1}{\lambda_1 + \lambda_2}(x - 1) + a + b;  
        \ \frac{1}{2} \le x \le 1\\
    \end{cases} 
\end{equation}

Из (\ref{e4}) и (\ref{e9}) видно, что $\stackrel{o}{T}(x) \equiv T(x)$ при условиях (\ref{e1}).   


\begin{equation}\label{e10}
    \begin{cases}
        q^{\Rmnum{1}}(x) = -\frac{2a\lambda_2\lambda_1}{\lambda_1 + \lambda_2};
        \ 0 \le x \le \frac{1}{2}\\
        q^{\Rmnum{2}}(x) = -\frac{2a\lambda_1\lambda_2}{\lambda_1 + \lambda_2};
        \ \frac{1}{2} \le x \le 1\\
    \end{cases} 
\end{equation}

где $q(x)=-\lambda(x)T'(x)$ - тепловой поток.


\begin{equation} \label{e11}
    \stackrel{o}{q}(x) = k(\xi)T'_0(x)
\end{equation}

$\stackrel{o}{q}(x)$ - асимптотическое приближение к $q(x)$.

\begin{equation*}
    \begin{cases}
        \stackrel{o}{q}^{\Rmnum{1}}(x) = 
        -\lambda_1(\frac{2\lambda_2}{\lambda_1 + \lambda_2} - 1 + 1)a;
        \ 0 \le x \le \frac{1}{2}\\
        \stackrel{o}{q}^{\Rmnum{2}}(x) =
        -\lambda_2(\frac{2\lambda_1}{\lambda_1 + \lambda_2} - 1 + 1)a;
        \ \frac{1}{2} \le x \le 1\\
    \end{cases} 
\end{equation*}

\begin{equation}\label{e12}
    \begin{cases}
        \stackrel{o}{q}^{\Rmnum{1}}(x) = -\frac{2a\lambda_2\lambda_1}{\lambda_1 + \lambda_2};
        \ 0 \le x \le \frac{1}{2}\\
        \stackrel{o}{q}^{\Rmnum{2}}(x) = -\frac{2a\lambda_1\lambda_2}{\lambda_1 + \lambda_2};
        \ \frac{1}{2} \le x \le 1\\
    \end{cases} 
\end{equation}

Из (\ref{e10}) и (\ref{e12}) видно, что $\stackrel{o}{q}(x) \equiv q(x)$ при условиях (\ref{e1}).   

\section{Тоже самое, но тело разбито не на две, а на четыре части.}

\begin{equation} \label{e13}
    \begin{array}{c}
        \frac{d}{dx} \big( \lambda(x) \frac{dT(x)}{dx} \big) = 0, \\ \\
        \left. T(x) \right|_{x=0} = T(0), \left. T(x) \right|_{x=1} = T(1). \\ \\
        \varepsilon = 1
    \end{array} 
\end{equation}

\begin{equation*}
    \begin{cases}
        \lambda(x) = \lambda_1; \ 0 \le x \le \frac{1}{4}\\
        \lambda(x) = \lambda_2; \ \frac{1}{4} \le x \le \frac{1}{2}\\
        \lambda(x) = \lambda_1; \ \frac{1}{2} \le x \le \frac{3}{4}\\
        \lambda(x) = \lambda_2; \ \frac{3}{4} \le x \le 1\\
    \end{cases} 
\end{equation*}

\subsection{Аналитическое решение.}

\begin{equation*}
    \begin{cases}
        T^{\Rmnum{1}}(x) = \frac{C}{\lambda_1}x + T(0); \ 0 \le x \le \frac{1}{4}\\ 
        T^{\Rmnum{2}}(x) = \frac{C}{\lambda_1}(x-\frac{1}{2}) -
        \frac{C}{\lambda_1}(\frac{1}{4}) -
        \frac{C}{\lambda_2}(\frac{1}{4}) + T(1); \ \frac{1}{4} \le x \le \frac{1}{2}\\ 
        T^{\Rmnum{3}}(x) = \frac{C}{\lambda_1}(x-\frac{3}{4}) -
        \frac{C}{\lambda_2}(\frac{1}{4}) + T(1); \ \frac{1}{2} \le x \le \frac{3}{4}\\ 
        T^{\Rmnum{4}}(x) = \frac{C}{\lambda_2}(x-1) + T(1); \ \frac{3}{4} \le x \le 1\\ 
    \end{cases} 
\end{equation*}

\begin{equation*}
    T^{\Rmnum{1}}(1/4) = T^{\Rmnum{2}}(1/4)
\end{equation*}

\begin{equation*}
    \frac{C}{\lambda_1}\frac{1}{4} + T(0) = 
        \frac{C}{\lambda_1}(\frac{1}{4}-\frac{1}{2}) -
        \frac{C}{\lambda_1}(\frac{1}{4}) -
        \frac{C}{\lambda_2}(\frac{1}{4}) + T(1) 
\end{equation*}

\begin{equation*}
    C = \frac{T(1)-T(0)}{\frac{1/4 + 1/4}{\lambda_1} + \frac{1/4 + 1/4}{\lambda_2}} =
    2\frac{\lambda_1 \lambda_2}{\lambda_1 + \lambda_2}(T(1)-T(0))
\end{equation*}

Если $T(0) = b$ и $T(1) = a+b$, то

\begin{equation}\label{e14}
    C = 2a \frac{\lambda_1 \lambda_2}{\lambda_1 + \lambda_2}
\end{equation}

\begin{equation}\label{e15}
    \begin{cases}
        T^{\Rmnum{1}}(x) = \frac{2a\lambda_2}{\lambda_1 + \lambda_2}x + b; \ 0 \le x \le \frac{1}{4}\\ 
        T^{\Rmnum{2}}(x) = \frac{2a\lambda_1}{\lambda_1 + \lambda_2}(x-\frac{1}{2}) -
        \frac{2a\lambda_2}{\lambda_1 + \lambda_2}(\frac{1}{4}) -
        \frac{2a\lambda_1}{\lambda_1 + \lambda_2}(\frac{1}{4}) + a + b; \ \frac{1}{4} \le x \le \frac{1}{2}\\ 
        T^{\Rmnum{3}}(x) = \frac{2a\lambda_2}{\lambda_1 + \lambda_2}(x-\frac{3}{4}) -
        \frac{2a\lambda_1}{\lambda_1 + \lambda_2}(\frac{1}{4}) + a + b; \ \frac{1}{2} \le x \le \frac{3}{4}\\ 
        T^{\Rmnum{4}}(x) = \frac{2a\lambda_1}{\lambda_1 + \lambda_2}(x-1) + a + b; \ \frac{3}{4} \le x \le 1\\ 
    \end{cases} 
\end{equation}

\begin{equation}\label{e16}
    \begin{cases}
        q^{\Rmnum{1}}(x) = -\frac{2a\lambda_1\lambda_2}{\lambda_1 + \lambda_2}\varepsilon; \ 0 \le x \le \frac{1}{4}\\ 
        q^{\Rmnum{2}}(x) = -\frac{2a\lambda_1\lambda_2}{\lambda_1 + \lambda_2}\varepsilon; \ \frac{1}{4} \le x \le \frac{1}{2}\\
        q^{\Rmnum{3}}(x) = -\frac{2a\lambda_1\lambda_2}{\lambda_1 + \lambda_2}\varepsilon; \ \frac{3}{4} \le x \le 1\\
        q^{\Rmnum{4}}(x) = -\frac{2a\lambda_1\lambda_2}{\lambda_1 + \lambda_2}\varepsilon; \ \frac{3}{4} \le x \le 1\\ 
    \end{cases} 
\end{equation}

\subsection{Асимптотическое решение.}

\begin{equation} \label{e17}
    \begin{array}{c}
        \frac{d}{d\xi} \big( \lambda(\xi) \frac{d\Psi(\xi)}{d\xi} \big) = 
        - \frac{d\lambda(\xi)}{d\xi}, \\ \\
        \left. \Psi(\xi) \right|_{\xi=0} = \left. \Psi(\xi) \right|_{\xi=1}. \\ \\
        \left. k(\xi) \right|_{\xi=0} = \left. k(\xi) \right|_{\xi=1}. \\ \\
        k(\xi) = -(\lambda(\xi) \frac{d\Psi(\xi)}{d\xi} + \frac{d\lambda(\xi)}{d\xi}) \\ \\
        \varepsilon = 1 \\ \\
        T(0) = 0
    \end{array} 
\end{equation}

\begin{equation*}
    \begin{cases}
        \Psi^{\Rmnum{1}}(\xi) = (\frac{C}{\lambda_1} - 1)\xi; \ 0 \le \xi \le \frac{1}{4}\\ 
        \Psi^{\Rmnum{2}}(\xi) = (\frac{C}{\lambda_1} - 1)(\xi-\frac{1}{2}) -
        (\frac{C}{\lambda_1} - 1)(\frac{1}{4}) -
        (\frac{C}{\lambda_2} - 1)(\frac{1}{4}); \ \frac{1}{4} \le \xi \le \frac{1}{2}\\ 
        \Psi^{\Rmnum{3}}(\xi) = (\frac{C}{\lambda_1} - 1)(\xi-\frac{3}{4}) -
        (\frac{C}{\lambda_2} - 1)(\frac{1}{4}); \ \frac{1}{2} \le \xi \le \frac{3}{4}\\ 
        \Psi^{\Rmnum{4}}(\xi) = (\frac{C}{\lambda_2} - 1)(\xi-1); \ \frac{3}{4} \le \xi \le 1\\ 
    \end{cases} 
\end{equation*}

\begin{equation*}
    \Psi^{\Rmnum{1}}(1/4) = \Psi^{\Rmnum{2}}(1/4)
\end{equation*}

\begin{equation*}
    (\frac{C}{\lambda_1} - 1)\frac{1}{4} = 
        (\frac{C}{\lambda_1} - 1)(\frac{1}{4}-\frac{1}{2}) -
        (\frac{C}{\lambda_1} - 1)(\frac{1}{4}) -
        (\frac{C}{\lambda_2} - 1)(\frac{1}{4}) 
\end{equation*}

\begin{equation}\label{e18}
    C = 2 \frac{\lambda_1 \lambda_2}{\lambda_1 + \lambda_2}
\end{equation}

\begin{equation} \label{e19}
    \begin{cases}
        \Psi^{\Rmnum{1}}(\xi) = (\frac{2\lambda_2}{\lambda_1 + \lambda_2} - 1)\xi; \ 0 \le \xi \le \frac{1}{4}\\ 
        \Psi^{\Rmnum{2}}(\xi) = (\frac{2\lambda_1}{\lambda_1 + \lambda_2}  - 1)(\xi-\frac{1}{2}) -
        (\frac{2\lambda_2}{\lambda_1 + \lambda_2} - 1)(\frac{1}{4}) -
        (\frac{2\lambda_1}{\lambda_1 + \lambda_2} - 1)(\frac{1}{4}); \ \frac{1}{4} \le \xi \le \frac{1}{2}\\ 
        \Psi^{\Rmnum{3}}(\xi) = (\frac{2\lambda_2}{\lambda_1 + \lambda_2} - 1)(\xi-\frac{3}{4}) -
        (\frac{2\lambda_1}{\lambda_1 + \lambda_2} - 1)(\frac{1}{4}); \ \frac{1}{2} \le \xi \le \frac{3}{4}\\ 
        \Psi^{\Rmnum{4}}(\xi) = (\frac{2\lambda_1}{\lambda_1 + \lambda_2}  - 1)(\xi-1); \ \frac{3}{4} \le \xi \le 1\\ 
    \end{cases} 
\end{equation}

\begin{equation*}
    \begin{cases}
        \stackrel{o}{T^{\Rmnum{1}}}(x) = 
        ax+b + a(\frac{2\lambda_2}{\lambda_1 + \lambda_2} - 1)x; \ 0 \le x \le \frac{1}{4}\\ 
        \stackrel{o}{T^{\Rmnum{2}}}(x) = 
        ax+b + a(\frac{2\lambda_1}{\lambda_1 + \lambda_2}  - 1)(x-\frac{1}{2}) -
        a(\frac{2\lambda_2}{\lambda_1 + \lambda_2} - 1)(\frac{1}{4}) -
        a(\frac{2\lambda_1}{\lambda_1 + \lambda_2} - 1)(\frac{1}{4}); \ \frac{1}{4} \le x \le \frac{1}{2}\\ 
        \stackrel{o}{T^{\Rmnum{3}}}(x) = 
        ax+b + a(\frac{2\lambda_2}{\lambda_1 + \lambda_2} - 1)(x-\frac{3}{4}) -
        a(\frac{2\lambda_1}{\lambda_1 + \lambda_2} - 1)(\frac{1}{4}); \ \frac{1}{2} \le x \le \frac{3}{4}\\ 
        \stackrel{o}{T^{\Rmnum{4}}}(x) = 
        ax+b + a(\frac{2\lambda_1}{\lambda_1 + \lambda_2}  - 1)(x-1); \ \frac{3}{4} \le x \le 1\\ 
    \end{cases} 
\end{equation*}

\begin{equation} \label{e20}
    \begin{cases}
        \stackrel{o}{T^{\Rmnum{1}}}(x) = 
        \frac{2a\lambda_2}{\lambda_1 + \lambda_2}x + b; \ 0 \le x \le \frac{1}{4}\\ 

        \stackrel{o}{T^{\Rmnum{2}}}(x) = 
        \frac{2a\lambda_1}{\lambda_1 + \lambda_2}(x-\frac{1}{2}) -
        \frac{2a\lambda_2}{\lambda_1 + \lambda_2}\frac{1}{4} -
        \frac{2a\lambda_1}{\lambda_1 + \lambda_2}\frac{1}{4} + a + b; \ \frac{1}{4} \le x \le \frac{1}{2}\\ 

        \stackrel{o}{T^{\Rmnum{3}}}(x) = 
        \frac{2a\lambda_2}{\lambda_1 + \lambda_2}(x-\frac{3}{4}) -
        \frac{2a\lambda_1}{\lambda_1 + \lambda_2}\frac{1}{4} + a + b; \ \frac{1}{2} \le x \le \frac{3}{4}\\ 
        
        \stackrel{o}{T^{\Rmnum{4}}}(x) = 
        \frac{2a\lambda_1}{\lambda_1 + \lambda_2}(x-1) + a + b; \ \frac{3}{4} \le x \le 1\\ 
    \end{cases} 
\end{equation}

\begin{equation*}
    \stackrel{o}{T}(x) \equiv T(x)
\end{equation*}

\begin{equation}\label{e21}
    \begin{cases}
        \stackrel{o}{q}^{\Rmnum{1}}(x) = -\frac{2a\lambda_1\lambda_2}{\lambda_1 + \lambda_2}; \ 0 \le x \le \frac{1}{4}\\ 
        \stackrel{o}{q}^{\Rmnum{2}}(x) = -\frac{2a\lambda_1\lambda_2}{\lambda_1 + \lambda_2}; \ \frac{1}{4} \le x \le \frac{1}{2}\\
        \stackrel{o}{q}^{\Rmnum{3}}(x) = -\frac{2a\lambda_1\lambda_2}{\lambda_1 + \lambda_2}; \ \frac{3}{4} \le x \le 1\\
        \stackrel{o}{q}^{\Rmnum{4}}(x) = -\frac{2a\lambda_1\lambda_2}{\lambda_1 + \lambda_2}; \ \frac{3}{4} \le x \le 1\\ 
    \end{cases} 
\end{equation}

\begin{equation*}
    \stackrel{o}{q}(x) \equiv q(x)
\end{equation*}

\section{Тоже самое, только теперь $\varepsilon = 1/2$.}

Аналитическое решение остаётся тем же, а асимптотическое меняется.

\begin{equation} \label{e22}
    \begin{cases}
        \Psi^{\Rmnum{1}}(\xi) = (\frac{2\lambda_2}{\lambda_1 + \lambda_2} - 1)\xi;
        \ 0 \le \xi \le \frac{1}{2}\\
        \Psi^{\Rmnum{2}}(\xi) = (\frac{2\lambda_1}{\lambda_1 + \lambda_2} - 1)(\xi - 1);
        \ \frac{1}{2} \le \xi \le 1\\
    \end{cases} 
\end{equation}

\begin{equation*}
    \begin{cases}
        \stackrel{o}{T^{\Rmnum{1}}}(x) = 
        T_0(x) + \Psi^{\Rmnum{1}}(\xi)T'_0(x)\varepsilon; \ 0 \le x \le \frac{1}{4}\\ 
        \stackrel{o}{T^{\Rmnum{2}}}(x) = 
        T_0(x) + \Psi^{\Rmnum{2}}(\xi)T'_0(x)\varepsilon; \ \frac{1}{4} \le x \le \frac{1}{2}\\ 
        \stackrel{o}{T^{\Rmnum{3}}}(x) = 
        T_0(x) + \Psi^{\Rmnum{1}}(\xi)T'_0(x)\varepsilon; \ \frac{1}{2} \le x \le \frac{3}{4}\\ 
        \stackrel{o}{T^{\Rmnum{4}}}(x) = 
        T_0(x) + \Psi^{\Rmnum{2}}(\xi)T'_0(x)\varepsilon; \ \frac{3}{4} \le x \le 1\\ 
    \end{cases} 
\end{equation*}

\begin{equation*}
        \stackrel{o}{T^{\Rmnum{1}}}(x) = 
        a + b + (\frac{2\lambda_2}{\lambda_1 + \lambda_2} - 1)\xi a\varepsilon
\end{equation*}

\begin{equation*}
    x = x_i + \xi \varepsilon, x_i = 0, \xi = \frac{x}{\varepsilon}
\end{equation*}

\begin{equation*}
        \stackrel{o}{T^{\Rmnum{1}}}(x) = 
        a + b + (\frac{2\lambda_2}{\lambda_1 + \lambda_2} - 1)\frac{x}{\varepsilon} a\varepsilon
\end{equation*}

\begin{equation*}
        \stackrel{o}{T^{\Rmnum{1}}}(x) = 
        \frac{2a\lambda_2}{\lambda_1 + \lambda_2}x + b;
\end{equation*}

\begin{equation*}
        \stackrel{o}{T^{\Rmnum{2}}}(x) = 
        a + b + (\frac{2\lambda_2}{\lambda_1 + \lambda_2} - 1)(\xi - 1) a\varepsilon
\end{equation*}

\begin{equation*}
    x = x_i + \xi \varepsilon, x_i = 0, \xi = \frac{x}{\varepsilon}
\end{equation*}

\begin{equation*}
    \begin{gathered}
        \stackrel{o}{T^{\Rmnum{2}}}(x) = 
        a + b + (\frac{2\lambda_2}{\lambda_1 + \lambda_2} - 1)
        (\frac{x}{\varepsilon}- 1) a\varepsilon = \\
        \frac{2\lambda_1}{\lambda_1 + \lambda_2}ax-
        (\frac{2\lambda_1}{\lambda_1 + \lambda_2}-1)a\varepsilon + b = \\
        \frac{2\lambda_1}{\lambda_1 + \lambda_2}ax-
        \frac{\lambda_1-\lambda_2}{2(\lambda_1+\lambda_2)}a + b
    \end{gathered}
\end{equation*}

\begin{equation*}
        \stackrel{o}{T^{\Rmnum{3}}}(x) = 
        a + b + (\frac{2\lambda_2}{\lambda_1 + \lambda_2} - 1)\xi a\varepsilon
\end{equation*}

\begin{equation*}
    x = x_i + \xi \varepsilon, x_i = 1/2, \xi = \frac{x-1/2}{\varepsilon}
\end{equation*}

\begin{equation*}
    \begin{gathered}
        \stackrel{o}{T^{\Rmnum{3}}}(x) = 
        a + b + (\frac{2\lambda_2}{\lambda_1 + \lambda_2} - 1)
        \frac{x-1/2}{\varepsilon} a\varepsilon = \\
        \frac{2\lambda_2}{\lambda_1 + \lambda_2}ax-
        \frac{\lambda_2-\lambda_1}{2(\lambda_1+\lambda_2)}a + b
    \end{gathered}
\end{equation*}

\begin{equation*}
        \stackrel{o}{T^{\Rmnum{4}}}(x) = 
        a + b + (\frac{2\lambda_2}{\lambda_1 + \lambda_2} - 1)(\xi-1) a\varepsilon
\end{equation*}

\begin{equation*}
    x = x_i + \xi \varepsilon, x_i = 1/2, \xi = \frac{x-1/2}{\varepsilon}
\end{equation*}

\begin{equation*}
    \begin{gathered}
        \stackrel{o}{T^{\Rmnum{4}}}(x) = 
        a + b + (\frac{2\lambda_2}{\lambda_1 + \lambda_2} - 1)
        (\frac{x-1/2}{\varepsilon}-1) a\varepsilon = \\
        \frac{2a\lambda_1}{\lambda_1 + \lambda_2}(x-1) + a + b; 
    \end{gathered}
\end{equation*}

\begin{equation} \label{e23}
    \begin{cases}
        \stackrel{o}{T^{\Rmnum{1}}}(x) = 
        \frac{2a\lambda_2}{\lambda_1 + \lambda_2}x + b; \ 0 \le x \le \frac{1}{4}\\ 

        \stackrel{o}{T^{\Rmnum{2}}}(x) = 
        \frac{2\lambda_1}{\lambda_1 + \lambda_2}ax-
        \frac{\lambda_1-\lambda_2}{2(\lambda_1+\lambda_2)}a + b
        ; \ \frac{1}{4} \le x \le \frac{1}{2}\\ 

        \stackrel{o}{T^{\Rmnum{3}}}(x) = 
        \frac{2\lambda_2}{\lambda_1 + \lambda_2}ax-
        \frac{\lambda_2-\lambda_1}{2(\lambda_1+\lambda_2)}a + b;
        \ \frac{1}{2} \le x \le \frac{3}{4}\\ 
        
        \stackrel{o}{T^{\Rmnum{4}}}(x) = 
        \frac{2a\lambda_1}{\lambda_1 + \lambda_2}(x-1) + a + b; \ \frac{3}{4} \le x \le 1\\ 
    \end{cases} 
\end{equation}

Преабразуем выражениия (\ref{e15}).

\begin{equation*}
    \begin{gathered}
        T^{\Rmnum{2}}(x) = \frac{2a\lambda_1}{\lambda_1 + \lambda_2}(x-\frac{1}{2}) -
        \frac{2\lambda_2}{\lambda_1 + \lambda_2}(\frac{1}{4}) -
        \frac{2\lambda_1}{\lambda_1 + \lambda_2}(\frac{1}{4}) + a + b = \\
        \frac{2\lambda_1}{\lambda_1 + \lambda_2}ax + 
        \frac{2\lambda_1}{\lambda_1 + \lambda_2}(\frac{1}{4}-\frac{1}{2})a + 
        \frac{2\lambda_2}{\lambda_1 + \lambda_2}(\frac{1}{4})a + a + b = \\
        \frac{2\lambda_1}{\lambda_1 + \lambda_2}ax - 
        \frac{\lambda_1-\lambda_2}{2(\lambda_1+\lambda_2)}a + b
    \end{gathered}
\end{equation*}

\begin{equation*}
    \begin{gathered}
        T^{\Rmnum{3}}(x) = \frac{2a\lambda_2}{\lambda_1 + \lambda_2}(x-\frac{3}{4}) -
        \frac{2\lambda_1}{\lambda_1 + \lambda_2}(\frac{1}{4}) + a + b = \\ 
        \frac{2\lambda_2}{\lambda_1 + \lambda_2}ax + 
        \frac{2\lambda_1}{\lambda_1 + \lambda_2}(\frac{3}{4}-1)a - 
        \frac{2\lambda_2}{\lambda_1 + \lambda_2}(\frac{3}{4})a + a + b = \\
        \frac{2\lambda_2}{\lambda_1 + \lambda_2}ax - 
        \frac{\lambda_2-\lambda_1}{2(\lambda_1+\lambda_2)}a + b
    \end{gathered}
\end{equation*}

Получили:

\begin{equation} \label{e24}
    \begin{cases}
        T^{\Rmnum{1}}(x) = 
        \frac{2a\lambda_2}{\lambda_1 + \lambda_2}x + b; \ 0 \le x \le \frac{1}{4}\\ 

        T^{\Rmnum{2}}(x) = 
        \frac{2\lambda_1}{\lambda_1 + \lambda_2}ax-
        \frac{\lambda_1-\lambda_2}{2(\lambda_1+\lambda_2)}a + b
        ; \ \frac{1}{4} \le x \le \frac{1}{2}\\ 

        T^{\Rmnum{3}}(x) = 
        \frac{2\lambda_2}{\lambda_1 + \lambda_2}ax-
        \frac{\lambda_2-\lambda_1}{2(\lambda_1+\lambda_2)}a + b;
        \ \frac{1}{2} \le x \le \frac{3}{4}\\ 
        
        T^{\Rmnum{4}}(x) = 
        \frac{2a\lambda_1}{\lambda_1 + \lambda_2}(x-1) + a + b; \ \frac{3}{4} \le x \le 1\\ 
    \end{cases} 
\end{equation}

Из (\ref{e23}) и (\ref{e24}) видно, что $\stackrel{o}{T}(x) \equiv T(x)$.   

\begin{equation*}
    \begin{cases}
        \stackrel{o}{q^{\Rmnum{1}}}(x) = 
        k^{\Rmnum{1}}(\xi)T'_0(x)\varepsilon; \ 0 \le x \le \frac{1}{4}\\ 
        \stackrel{o}{q^{\Rmnum{2}}}(x) = 
        k^{\Rmnum{2}}(\xi)T'_0(x)\varepsilon; \ \frac{1}{4} \le x \le \frac{1}{2}\\ 
        \stackrel{o}{q^{\Rmnum{3}}}(x) = 
        k^{\Rmnum{1}}(\xi)T'_0(x)\varepsilon; \ \frac{1}{2} \le x \le \frac{3}{4}\\ 
        \stackrel{o}{q^{\Rmnum{4}}}(x) = 
        k^{\Rmnum{2}}(\xi)T'_0(x)\varepsilon; \ \frac{3}{4} \le x \le 1\\ 
    \end{cases} 
\end{equation*}

\begin{equation*}
    \begin{cases}
        \stackrel{o}{q}^{\Rmnum{1}}(x) = 
        -\lambda_1(\frac{2\lambda_2}{\lambda_1 + \lambda_2} - 1 + 1)a\varepsilon;
        \ 0 \le x \le \frac{1}{4}\\ 
        \stackrel{o}{q}^{\Rmnum{2}}(x) = 
        -\lambda_2(\frac{2\lambda_1}{\lambda_1 + \lambda_2} - 1 + 1)a\varepsilon;
        \ \frac{1}{4} \le x \le \frac{1}{2}\\ 
        \stackrel{o}{q}^{\Rmnum{3}}(x) = 
        -\lambda_1(\frac{2\lambda_2}{\lambda_1 + \lambda_2} - 1 + 1)a\varepsilon;
        \ \frac{1}{2} \le x \le \frac{3}{4}\\ 
        \stackrel{o}{q}^{\Rmnum{4}}(x) = 
        -\lambda_2(\frac{2\lambda_1}{\lambda_1 + \lambda_2} - 1 + 1)a\varepsilon;
        \ \frac{3}{4} \le x \le 1\\ 
    \end{cases} 
\end{equation*}

\begin{equation} \label{e25}
    \begin{cases}
        \stackrel{o}{q}^{\Rmnum{1}}(x) = 
        -\frac{2\lambda_1\lambda_2}{\lambda_1 + \lambda_2}a\varepsilon;
        \ 0 \le x \le \frac{1}{4}\\ 
        \stackrel{o}{q}^{\Rmnum{2}}(x) = 
        -\frac{2\lambda_1\lambda_2}{\lambda_1 + \lambda_2}a\varepsilon;
        \ \frac{1}{4} \le x \le \frac{1}{2}\\ 
        \stackrel{o}{q}^{\Rmnum{3}}(x) = 
        -\frac{2\lambda_1\lambda_2}{\lambda_1 + \lambda_2}a\varepsilon;
        \ \frac{1}{2} \le x \le \frac{3}{4}\\ 
        \stackrel{o}{q}^{\Rmnum{4}}(x) = 
        -\frac{2\lambda_1\lambda_2}{\lambda_1 + \lambda_2}a\varepsilon;
        \ \frac{3}{4} \le x \le 1\\ 
    \end{cases} 
\end{equation}

Из (\ref{e16}) и (\ref{e25}) видно, что $\stackrel{o}{q}(x) \equiv q(x)$.   


\end{document}













