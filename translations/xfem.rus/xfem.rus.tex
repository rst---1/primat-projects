\documentclass[a4paper,12pt]{article}
\usepackage{graphicx}
\usepackage[utf8]{inputenc}
\usepackage[russian]{babel}
\usepackage{amsmath,amssymb}
\usepackage{tikz}
\usepackage{marvosym}
\usepackage{pgfplots}
\pgfplotsset{compat=1.7}
\begin{document}

\title{Метод конечных элементов для моделирования роста трещины без изменения расчетной сетки}

\maketitle

Совершенствование новой техники для моделирования трещин в рамках
конечного элемента представлен.Стандартное displacement-based аппроксимации
на основе обогащен рядом с трещиной путем включения и разрывными полях и
вблизи вершины асимптотических полей через перегородку единство метода. 
Методологию, которая создает обогащенный приближение при взаимодействии
геометрии трещины с сеткой разработана. Этот метод позволяет всей трещины
должны быть представлены независимо от сетки, и поэтому перестройку сетки
не является необходимым для моделирования роста трещин. Численные эксперименты
предоставляются чтобы продемонстрировать полезность и надежность предлагаемой
техники.


\section{Введение.}

Моделирование движущихся разрывов с методом конечных элементов является
громоздким из-за необходимости обновления сетки в соответствии с геометрией разрыва.
Несколько новых методов конечных элементов были разработаны для модели трещины и
трещин без перестройку сетки. К ним относятся включение прерывистого режима на
уровне элемента [1], движущейся техникой сетки [2], и обогащение техники для
конечных элементов на основе раздела-из-единство, которое включает в себя
минимальный перестройку сетки [3].

В Belytschko и Black [3], изогнутые трещин обрабатывали путем отображения
прямой трещины обогащенный области. Это не легко применимы к длинных трещин
или трех измерениях. В этой статье мы совершенствования метода путем включения
разрывного поля через трещины обращена в сторону от трещины. Метод включает в
себя как разрывная функция Хаара и вблизи иглы асимптотической функции через
разбиение единицы методом. Трещина выросла на пересмотр кончик расположения и
добавления новых сегментов трещину. Кроме того разрывной поле позволяет в течение
всего геометрии трещины должны быть смоделированы независимо от сетки, а также
полностью исключает необходимость перерасчета как трещины возрастает.

Настоящая методика использует свойство разбиения единицы конечных элементов
определены Melenk и Babushka [4], что позволяет местным функций обогащению быть
легко включены в конечно-элементной аппроксимации. Стандартного приближения,
таким образом, «обогащенный» в представляющей интерес области местной функций в
сочетании с дополнительной степени свободы. С целью анализа трещиноватости,
обогащение функции вблизи иглы асимптотических полей и разрывной функцией
представлять скачок перемещение через трещины линии.

Метод отличается от работы Oliver [5], который вводит ступенчатых функций в
поле смещения, а затем рассматривает воздействие на уровне элемента многопрофильный
подход с предполагаемым полем деформации. В способе, описанном в этой статье,
смещение поле действительно глобальные, но поддержка обогащения функций местных,
потому что они умножаются на функции узловой формы.

Эта статья организована следующим образом. В следующем разделе мы представляем
сильные и слабые формы линейной упругой механики разрушения.Дискретных уравнений
приведены в разделе 3, который также описывает включение обогащению функций для
моделирования трещин. Несколько численные примеры приведены в разделе 4. Наконец,
в разделе 5 приводится краткая информация и некоторые заключительные замечания.


\section{Формулировка проблемы.}

В этом разделе мы кратко рассмотрим основные уравнения для упруго-статики
и дать соответствующие слабой форме. В частности, мы рассмотрим случай,
когда внутренняя линия присутствует, через которую поля смещения может быть прерывистым.

Рассмотрим область $\Omega$ с гранницей $\Gamma$. Гранница $\Gamma$ состоит из
композиции границ $\Gamma_u$, $\Gamma_t$, $\Gamma_c$, так что 
$\Gamma = \Gamma_u \cup \Gamma_t \cup \Gamma_c$, как показанно на рис.1. 
Внешние силы прикладываютя к границе $Gamma_t$, а граница $Gamma_u$ закреплена.
На поверхности $Gamma_c$ (кривая в двумерном пространстве и поверхность в трехмерном)
заданно условие свободной граници.
Уравнение равновесия и граничные условия:

где n - внешняя нопрмаль, $\sigma$ - напряжения Коши, b - обьемные силы.

В данном исследовании мы рассматриваем малые
деформации и смещений. Кинематика уравнений, следовательно,
состоять из штамма смещения отношения:

\begin{equation}
    \varepsilon = \varepsilon(u) = \nabla_su
\end{equation} 

здесь $\nabla_s$ симметричная часть градиента. 
На границе:
\begin{equation}
    u = \overline{u}, \Gamma_u
\end{equation} 

Напряжения определяются через дефолрмации по закону Гука:

\begin{equation}
    \sigma = C : \varepsilon
\end{equation} 

где C - тензор Гука.

\section{Слабая форма}
% \begin{equation}
%     \begin{array}{c}
%         {\sigma_{\alpha\beta}} = \sum\limits_{\varphi \in \{x,y,z\}}
%         \sum\limits_{k_x+k_y+k_z=1}\left( 
%         \left(\tau_{\alpha\beta}^{v_{\varphi}}\right)^{\bar{k}}
%         \frac{\partial v_{\varphi}}{\partial \bar{r}^{\bar{k}}}
%     \varepsilon \right) = \\ \\
%     = \left(\tau_{\alpha\beta}^{xx} \frac{\partial v_{x}}{\partial x} +
%     \tau_{\alpha\beta}^{xy} \frac{\partial v_{x}}{\partial y} +
%     \tau_{\alpha\beta}^{xz} \frac{\partial v_{x}}{\partial z} +
%     \tau_{\alpha\beta}^{yx} \frac{\partial v_{y}}{\partial x} +
%     \tau_{\alpha\beta}^{yy} \frac{\partial v_{y}}{\partial y} +
%     \tau_{\alpha\beta}^{yz} \frac{\partial v_{y}}{\partial z} +
%     \tau_{\alpha\beta}^{zx} \frac{\partial v_{z}}{\partial x} +
%     \tau_{\alpha\beta}^{zy} \frac{\partial v_{z}}{\partial y} +
%     \tau_{\alpha\beta}^{zz} \frac{\partial v_{z}}{\partial z}\right)\varepsilon = \\ \\
%     = \left(\tau_{\alpha\beta}^{xx} \frac{\partial v_{x}}{\partial x} +
%     \tau_{\alpha\beta}^{yy} \frac{\partial v_{y}}{\partial y} +
%     \tau_{\alpha\beta}^{zz} \frac{\partial v_{z}}{\partial z}\right)\varepsilon 
%     \end{array} 
% \end{equation} 
% 
% \begin{equation}
%     \begin{array}{c}
%         \frac{\partial v_{x}}{\partial x} =
%         \frac{1}{\widetilde{E}_{xx}}\widetilde{\sigma}_{xx}; \\ \\
%         \frac{\partial v_{y}}{\partial y} = -
%         \frac{\widetilde{\nu}_{yx}}{\widetilde{E}_{xx}}\widetilde{\sigma}_{xx}; \\ \\
%         \frac{\partial v_{z}}{\partial z} = -
%         \frac{\widetilde{\nu}_{zx}}{\widetilde{E}_{xx}}\widetilde{\sigma}_{xx}
%     \end{array} 
% \end{equation} 
% 
% \begin{equation}
%     {\sigma_{\alpha\beta}} = \frac{\widetilde{\sigma}_{xx}\varepsilon}{\widetilde{E}_{xx}}
%     \left(\tau_{\alpha\beta}^{xx} -
%     \widetilde{\nu}_{yx} \tau_{\alpha\beta}^{yy} -
%     \widetilde{\nu}_{zx} \tau_{\alpha\beta}^{zz}\right)
% \end{equation} 

\end{document}













